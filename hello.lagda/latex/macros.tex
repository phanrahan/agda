\newcommand\nc\newcommand
\nc\rnc\renewcommand

\input{agda-commands}
%% Unicode character translations

\newcommand\uni[2]{\newunicodechar{#1}{\ensuremath{#2}}}
%% Remember to express horizontal movement in ems and vertical in exs and
%% nothing in pt, to adapt to font size changes.

\uni{⦃}{\{\hspace{-0.41em}\{}
\uni{⦄}{\}\hspace{-0.41em}\}}

\newcommand\sm[1]{\mathsf{#1}}

\usepackage{bm}
\uni{ℕ}{\mathbb{N}}
\uni{𝔹}{\mathbb{B}}
\uni{𝕥}{\sm{t}}
\uni{𝕗}{\sm{f}}

%% \usepackage{wasysym}

%% \uni{𝐶}{\mathit{C}}

%% \uni{μ}{\mu}
\uni{⊗}{\otimes}
\uni{⊎}{\uplus}
\uni{×}{\times}
\uni{∘}{\circ}
\uni{▵}{\vartriangle}
\uni{∎}{\blacksquare}
\uni{■}{\blacksquare}
\uni{∙}{\bullet}
\uni{□}{\bigbox}
\uni{⋯}{\cdots}
\uni{∷}{\mathbin{::}}
%% \uni{;}{;}

%% \uni{≅}{\cong}
%% \uni{𝟎}{\mathbf{0}}

%% \usepackage{marvosym}
%% \uni{✄}{\Rightscissors}

% \Rightarrow (⇒)
% \rightarrowtail (↣)
% \rightarrowtriangle

\usepackage{stmaryrd}

\uni{⇨}{\rightarrowtriangle}
\uni{⇧}{\uparrow}
%% \uni{↑}{\uparrow}
%% %% \uni{↑}{\hspace{-1pt}{^{\scriptscriptstyle\uparrow}}\hspace{-1pt}}
%% \uni{⇑}{\Uparrow}
%% \uni{⟰}{\dot\Uparrow}
%% \uni{⇉}{\rightrightarrows}
%% \uni{↦}{\rightarrowtail}
%% %% \Rightarrow \rightarrowtail
\uni{⟲}{\circlearrowleft}
%% \uni{↻}{\curvearrowleft}
\uni{↫}{\looparrowleft}
\uni{↰}{\Lsh}
\uni{⇋}{\rightleftharpoons}

\uni{Θ}{\Theta}
%% \uni{α}{\alpha}
%% \uni{β}{\beta}
%% \uni{γ}{\gamma}
\uni{λ}{\lambda}
%% \uni{μ}{\mu}
%% \uni{τ}{\tau}
%% \uni{ρ}{\rho}
%% \uni{σ}{\sigma}

%% \uni{ᶜ}{^c}
%% \uni{ᵀ}{^T}
%% \uni{≗}{\ \circeq\ }
%% \uni{·}{\cdot}
%% \uni{◎}{\circledcirc}
\uni{⊕}{\oplus}
\uni{∧}{\wedge}
\uni{∨}{\vee}
\uni{⊤}{\top}
%% \uni{⊥}{\bot}
%% \uni{✢}{+}

%% Subscripts
\nc\sub[1]{_{\scriptscriptstyle #1}}

\uni{₀}{\sub 0}
\uni{₁}{\sub 1}
\uni{₂}{\sub 2}
\uni{₃}{\sub 3}
\uni{₄}{\sub 4}
\uni{₅}{\sub 5}
\uni{₆}{\sub 6}
\uni{₇}{\sub 7}
\uni{₈}{\sub 8}
\uni{₉}{\sub 9}

\uni{ᵢ}{_i}
\uni{ₒ}{_o}
%% \uni{ₘ}{_m}
\uni{ₙ}{_n}
%% \uni{ᵣ}{_r}
%% \uni{ₛ}{_s}

%% %% Superscripts
%% \uni{²}{^2}
%% \uni{³}{^3}
%% \uni{⁻}{\raisebox{0.5ex}{-}}
\uni{ᵉ}{\raisebox{0.5ex}{\smaller e}}
\uni{ⁱ}{\raisebox{0.5ex}{\smaller i}}
\uni{ᵒ}{\raisebox{0.5ex}{\smaller o}}
\uni{ˡ}{\raisebox{0.5ex}{\smaller[2]l}}
\uni{ʳ}{\raisebox{0.5ex}{\smaller r}}
\uni{ᵛ}{\raisebox{0.5ex}{\smaller v}}
%% \uni{ⱽ}{^v}
%% \uni{ᶻ}{^z}
\uni{ᵇ}{^b}

%% \uni{⊣}{\dashv}

%% \RequirePackage{bbding}

%% \uni{≈}{\approx}
\uni{≡}{\equiv}
\uni{≗}{\circeq}
\uni{∀}{\forall}
%% \uni{≤}{\le}
%% \uni{⊹}{\dotplus}

\uni{⟨}{\langle}
\uni{⟩}{\rangle}

%%https://tex.stackexchange.com/questions/486120/using-the-unicode-prime-character-outside-math-mode-with-unicode-math-and-newuni
\AtBeginDocument{\uni{′}{\ensuremath{'}}}
\uni{″}{\ensuremath{'\hspace{-1pt}'}}


%% \uni{✯}{\mathop{\mbox{\tiny \FiveStarOutlineHeavy}}}

%% Invisible characters for otherwise clashing definitions.
\uni{⇂}{}
\uni{⇃}{}

%% Token translations. See section "Controlling the typesetting of
%% individual tokens" in the Agda User Manual and
%% https://tex.stackexchange.com/questions/64131/implementing-switch-cases .

\uni{✴}{\ast}
\uni{✢}{\raisebox{0.75pt}{\scalebox{0.65}{\ensuremath{+}}}}

\usepackage{xstring}
\nc\ehat[1]{\ensuremath{\hat{#1}}}
\nc\edot[1]{\ensuremath{\dot{#1}}}
\nc\etilde[1]{\ensuremath{\tilde{#1}}}
\DeclareRobustCommand{\AgdaFormat}[2]{\IfStrEqCase{#1}{
  {+̇}{+}
  %% {+}{✢}
  {*}{✴}
  % {;̂}{\ehat{;}}
  {∘̂}{\ehat{∘}}
  {𝕥̇}{𝕥}
  {𝕗̇}{𝕗}
  %% {2*}{2✴}
  %% {2*\AgdaUnderscore{}}{2✴\AgdaUnderscore{}}
  %% %% {*-assocˡ}{✴-assocˡ}
  %% %% {*-assocʳ}{✴-assocʳ}
  %% %% {*-comm}{✴-comm}
  %% {Mᵒ}{M\raisebox{0.7ex}{\smaller o}}
  %% {Rᵒ}{R\raisebox{0.7ex}{\smaller o}}
  %% {Mⁱ}{M\raisebox{0.7ex}{\smaller i}}
  %% {Rⁱ}{R\raisebox{0.7ex}{\smaller i}}
  %% {fᴹ}{f\hspace{1pt}\raisebox{0.8ex}{\tiny M}}
  %% {fᴿ}{f\hspace{1pt}\raisebox{0.8ex}{\tiny R}}


{}{}}[#2]}


  %%{*}{\hspace{1.4pt}✴\hspace{1.4pt}}


\nc\out[1]{}

%% \nc\noteOut[2]{\note{#1}\out{#2}}

%% To redefine for a non-draft
\nc\indraft[1]{#1}

%% I think \note gets defined by beamer.
\let\note\undefined

\nc\note[1]{\indraft{\textcolor{red}{#1}}}

\nc\notefoot[1]{\note{\footnote{\note{#1}}}}

\nc\todo[1]{\note{To do: #1}}

\nc\chapterl[1]{\chapter{#1}\chaplabel{#1}}
\nc\chaplabel[1]{\label{chap:#1}}
\nc\chapref[1]{Chapter~\ref{chap:#1}}

\nc\needcite{\note{[ref]}}

\nc\workingHere{
\vspace{1ex}
\begin{center}
\setlength{\fboxsep}{3ex}
\setlength{\fboxrule}{4pt}
\huge\textcolor{red}{\framebox{Working here}}
\end{center}
\vspace{1ex}
}

%% Would be superfluous with the memoir document class except for notefoot.
%% For multiple footnotes at a point. Adapted to recognize \notefoot as well
%% as \footnote. See https://tex.stackexchange.com/a/71347,
\let\oldFootnote\footnote
\nc\nextToken\relax
\rnc\footnote[1]{%
    \oldFootnote{#1}\futurelet\nextToken\isFootnote}
\nc\footcomma[1]{\ifx#1\nextToken\textsuperscript{,}\fi}
\nc\isFootnote{%
    \footcomma\footnote
    \footcomma\notefoot
}

% Arguments: env, label, caption, body
\nc\figdefG[4]{\begin{#1}%[tbp]
\begin{center}
#4
\end{center}
\vspace{-1ex}
\caption{#3}
\figlabel{#2}
\end{#1}}

% Arguments: label, caption, body
\nc\figdef{\figdefG{figure}}
\nc\figdefwide{\figdefG{figure*}}

% Arguments: label, caption, body
\nc\figrefdef[3]{\figref{#1}\figdef{#1}{#2}{#3}}

\nc\chapname{None}
\nc\chap[1]{\rnc\chapname{#1}\subfile{\chapname/Chapter}}
%% Is there a standard macro/variable I can use instead?

\nc\verilog[1]{\inputminted{v}{Figures/\chapname/#1.v}}

\nc\picfile[1]{Figures/#1.pdf}

%% \nc\run[1]{\begin{snugshade*}\agda{#1-run}\end{snugshade*}}
\nc\run[1]{\agda{#1-run}}

\nc\figW[2]{\begin{center}\agda{#2}{\includegraphics[width=#1\linewidth]{\picfile{\chapname/#2}}}\end{center}}

\nc\fig{\figW{1}}

\nc\exampleW[2]{
\vspace{-2ex}
\begin{center}\small
\hfill
\begin{minipage}[c]{0.40\linewidth}
\figW{#1}{#2}
\end{minipage}
\hspace{0.01\linewidth}
\begin{minipage}[c]{0.57\linewidth}\verilog{#2}\end{minipage}
\end{center}
\begin{center}
\vspace{-1ex}
\run{#2}
\ \\[-8ex]
\end{center}
%% \begin{snugshade*}
%% \vspace{-1ex}
%% \run{#2}
%% \vspace{-1ex}
%% \end{snugshade*}
%% \begin{code}
%%
%% \end{code}
\par\noindent\hspace{-1ex}%
}
%% The \par here prevents a LaTeX stack overflow. Huh?

\setlength{\fboxrule}{0.2pt}

\nc\example{\exampleW{1}}

\nc\source{../\chapname/Code}

\nc\agda[1]{\ExecuteMetaData[\source.tex]{#1}}

%% Hide the somewhat arbitrary distinction between fundamental methods and
%% derived operations.
\rnc\AgdaField\AgdaFunction

\usepackage{hyperref}
